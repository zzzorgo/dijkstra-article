\documentclass[9pt, letterpaper, twocolumn]{article}
\usepackage{import}
\usepackage{cuted}

\usepackage{lipsum}

\import{preambule/}{graphics.tex}
\import{preambule/defaults/}{index.tex}
\import{preambule/}{listing.tex}
\import{preambule/}{titles.tex}

\import{commands/}{section-abstract.tex}
\import{commands/}{font-weight.tex}
\import{commands/}{rulled-element.tex}

\title{\LaTeX}
\date{February 2014}
\author{Zzzorgo \thanks{thanks to bro}}

\usepackage{subfiles}

% ========================== Content ========================== %

\begin{document}


\begin{titlepage}
    \maketitle
    \begin{abstract}
        Какая-то статья
    \end{abstract}
    \listoffigures
    \tableofcontents
\end{titlepage}

\section{Введение}
\subsection{Точное введение}

\subfile{sections/introduction}
\subfile{sections/tree}

This text uses {\myfont a different} font typeface привет

\sectionabstract{Hey}{
    Lorem ipsum dolor sit amet, consectetuer adipiscing elit. Ut
    purus elit, vestibulum ut, placerat ac, adipiscing vitae, felis.
    Curabitur dictum gravida mauris. Nam arcu libero, nonummy
    eget, consectetuer id, vulputate a, magna. Donec vehicula
    augue eu neque. Pellentesque habitant morbi tristique senectus
    et netus et malesuada fames ac turpis egestas. Mauris ut leo.
    Cras viverra metus rhoncus sem.
}

\fontweight{\mainfontname}{Black}{Lorem}

\lipsum[1-12]

\tikz \graph
[
    spring layout,
    nodes={circle,draw,ultra thick,backgroundColor,fill=secondaryAccentColor,minimum size=7mm,text=backgroundColor},
    node distance=1.9cm,
    visited/.style={fill=accentColor,text=backgroundColor},
    edges={shorten <=1pt,>={Stealth[round]},semithick,mainColor}
]
{
    abc [visited] -- { b, c, d, e -> {f, g, я} };
    я -- abc;
};

\end{document}
