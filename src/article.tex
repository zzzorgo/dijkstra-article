\documentclass[9pt, letterpaper, twocolumn]{article}
\usepackage{import}
\usepackage{subcaption}
\usepackage{float}
\usepackage{xargs}

\import{preambule/}{graphics.tex}
\import{preambule/defaults/}{index.tex}
\import{preambule/}{listing.tex}
\import{preambule/}{titles.tex}

\renewcommand{\thesubfigure}{\asbuk{subfigure}}

\import{commands/}{section-abstract.tex}
\import{commands/}{font-weight.tex}
\import{commands/}{rulled-element.tex}
\import{commands/}{hatch-pattern.tex}
\import{commands/}{twocolumn-page-break.tex}
\import{commands/}{styled-graph.tex}
\import{commands/}{http-link.tex}

\title{Алгоритм Дейкстры}
\date{\vspace{-5ex}}
% \author{Zzzorgo \thanks{thanks to bro}}

\usepackage{subfiles}

% ========================== Content ========================== %

\begin{document}

\begin{titlepage}
    \pagecolor{accentColor}
    \vspace*{5cm}
    {\titlefont \textcolor{backgroundColor}{Алгоритм Дейкстры}}
    \vspace{0.5cm}

    \textcolor{backgroundColor}{\Large Коротко  (ну вообще-то не очень) и ясно ищем кратчайший путь в графе}
    \vfill
    \textcolor{backgroundColor}{{\large © zzzorgo 2022}}
    \clearpage

    \pagecolor{backgroundColor}
    \vspace*{5cm}
    {\titlefont Алгоритм Дейкстры}
    \vspace{0.5cm}

    \Large Коротко  (ну вообще-то не очень) и ясно ищем кратчайший путь в графе
    \vfill
    {\large © zzzorgo 2022}

    \clearpage
    \newgeometry{textwidth=10cm, onecolumn, top=1.5cm, bottom=2.5cm}
    \pagecolor{backgroundColor}
    \vspace*{\fill}
    \tableofcontents
    \vspace*{\fill}
    \restoregeometry
    \clearpage
\end{titlepage}


\subfile{sections/introduction/introduction.tex}
\subfile{sections/tree/tree.tex}
\subfile{sections/graph/graph.tex}
\subfile{sections/dijkstra/dijkstra.tex}

\end{document}
