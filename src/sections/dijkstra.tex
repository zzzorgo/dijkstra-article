\documentclass[../article.tex]{subfiles}
\begin{document}

\sectionabstract{Алгоритм Дейкстры}
{Мы уже поняли как найти кратчайший путь в не взвешенном графе, но для взвешенного графа само понятие кратчайшего пути отличается, а для его поиска в очередной раз немного изменим нашу структуру и алгоритм.}

Взвешенный граф, это граф, ребра которого имеют вес, то есть некое число, ассоциированное с этим ребром. Кротчайшим путем называют путь, сумма весов всех ребер которого наименьшая из возможных. Иллюстрация этого на рисунке \ref{fig:compareWeightedPaths}.

\begin{figure}
    \styledgraph{
        a -> [edge label=7] b,
        b -> [edge label=11, inner sep=1pt] c,
        b -> [edge label=2] d,
        d -> [edge label=1, inner sep=1.5pt] e,
        f -> [edge label=7, inner sep=1.5pt] g,
        g -> [edge label=4, near start] h,
        g -> [edge label=8] i,
        a -> [edge label=5] f,
        i -> [bend right=45, edge label=10] f,
        c -> [bend left=45, edge label=5] a,
        h <- [edge label=100] a,
    }
    \caption{Пример взвешенного графа}
    % \label{fig:compareWeightedPaths}
\end{figure}

\begin{figure}
    \styledgraph{
        a -> [disabledEdge] b [disabledNode],
        b -> [disabledEdge] c [disabledNode],
        b -> [disabledEdge] d [disabledNode],
        d -> [disabledEdge] e [disabledNode],
        f -> [edge label=7, inner sep=1.5pt] g,
        g -> [edge label=4, near start] h,
        g -> [disabledEdge] i [disabledNode],
        a -> [edge label=5] f,
        i -> [bend right=45, disabledEdge] f,
        c -> [bend left=45, disabledEdge] a,
        h <- [edge label=100] a,
    }
    \caption{Пути во взвешенном графе от {\firacodebold a} к {\firacodebold h}}
    \label{fig:compareWeightedPaths}
\end{figure}

И хотя длина пути {\firacodebold [a, h]} в количестве ребер меньше, чем у {\firacodebold [a, f, g, h]}, все таки во взвешенном графе путь {\firacodebold [a, f, g, h]} считается короче, так как длина пути {\firacodebold [a, h]} равна 100 и больше, чем длина пути {\firacodebold [a, f, g, h]}, равная 5 + 7 + 4 = 16.

При таком определении кратчайшего пути мы все еще можем использовать как поиск в глубину так и в ширину для нахождения решения. Для этого достаточно найти все пути и выбрать кротчайший. Или чуть проще – при обнаружении очередного целевого пути оставлять в памяти только тот, что короче.

Но прежде всего, надо в очередной раз поменять структуру данных, представляющую граф. Внимательный читатель заметил, что у нас появились веса ребер, но нет переменных, чтобы их хранить. До этого момента ребра нас не интересовали и были абсолютно равнозначны между собой, теперь же нам нужно выделить отдельную сущность под это (листинг \ref{lst:weighetGrahStruct}).

\subsection{Структура для обхода}

В этой структуре каждая вершина теперь имеет массив ребер {\firacodebold edges}, которые выходят из этой вершины, а каждое ребро {\firacodebold Edge} в этом массиве имеет вес {\firacodebold weight} и ссылку на вершину на другом конце ребра {\firacodebold nextNode}. Таким образом массив {\firacodebold neighbors} нам больше не нужен, информация о соседях перенесена в массив {\firacodebold edges}.

\begin{ruledelement}
    \lstset{
        emph={edges},
        emphstyle=\textbf,
    }
    \begin{lstlisting}[caption={Структура для взвешенного графа}, label={lst:weighetGrahStruct}]
interface Edge {
    weight: number;
    nextNode: Node;
}

interface Node {
    id: string;
    edges: Edge[];
    visited?: boolean;
    parent?: Node;
}
    \end{lstlisting}
\end{ruledelement}

\begin{figure*}
    \lstset{
        emph={nextEdge, edges},
        emphstyle=\textbf,
    }
    \begin{ruledelement}
        \begin{lstlisting}[caption={Поиск пути при обходе взвешенного графа в ширину}, label={lst:bfsWeightedRandomPath}]
function traverseGraph() {
    let currentNode = null;
    const nextEdges = [currentNode];

    while (nextNodes.length !== 0) {
        currentNode = nextEdges.shift();
        currentNode.visited = true;

        if (currentNode.data === 'h') {
            let pathNode = currentNode;
            const path = [];

            while(pathNode) {
                path.ushift(pathNode);
                pathNode = pathNode.parent;
            }

            return path;
        }

        const edgesToAdd = currentNode.edges
            .filter(nextEdge => !nextEdge.nextNode.visited)
            .map(nextEdge => {
                nextEdge.nextNode.parent = currentNode;

                return nextEdge.nextNode;
            });

        nextEdges.push(...edgesToAdd);
    }

    return null;
}
        \end{lstlisting}
    \end{ruledelement}

\end{figure*}

\begin{figure*}
    \lstset{
        emph={targetNode, newDistance},
        emphstyle=\textbf,
    }
    \begin{ruledelement}
        \begin{lstlisting}[caption={Первый подход к поиску кратчайшего пути}, label={lst:bfsShortestPath}]
function traverseGraph() {
    let currentNode = null;
    let targetNode = null;
    const nextNodes = [startingNode];
    // distance from `startingNode` to itself is zero
    const distances = new Map([[startingNode.id, 0]]);

    while (nextEdges.length !== 0) {
        currentNode = nextEdges.shift(); (*\label{code:bfsShortestPath:unshift}*)
        currentNode.visited = true;

        if (currentNode.data === 'h') {
            targetNode = currentNode;
        }

        for (const nextEdge of currentNode.edges) {
            const { nextNode } = nextEdge;
            const newDistance = distances.get(currentNode.id) + nextEdge.weight;

            // has another path to `nextNode`, need to select the shortest
            if (distances.has(nextNode.id)) {
                const knownDistance = distances.get(nextNode.id);

                // the new path is shorter, than the known
                if (knownDistance > newDistance) {
                    distances.set(nextNode.id, newDistance);
                    nextNode.parent = currentNode;
                }

            // the first known path to `nextNode`
            } else {
                distances.set(nextNode.id, newDistance);
                nextNode.parent = currentNode;
            }

            if (!nextEdge.nextNode.visited) {
                nextEdges.push(nextEdge.nextNode);
            }
        }
    }

    let pathNode = targetNode;
    const path = [];

    while(pathNode) {
        path.ushift(pathNode);
        pathNode = pathNode.parent;
    }

    return path;
}
        \end{lstlisting}
    \end{ruledelement}

\end{figure*}

\subsection{Обход взвешенного графа}

Конечно, это все тот же граф и концептуально его обход не изменился, но так как поменялась структура данных, то некоторые детали реализации тоже поменялись. Пример поиска произвольного (не обязательного кратчайшего) пути при обходе в ширину такой структуры показан на листинге \ref{lst:bfsWeightedRandomPath}.

\subsection{Поиск кратчайшего пути}

Теперь осталось лишь научится выбирать кратчайший путь по тому принципу, который был описан выше: запоминать первый найденный путь до целевой вершины и потом заменять его каждым следующим более коротким путем до целевой вершины.

Если подумать, то чтобы найти кратчайшее расстояние до целевой вершины, нужно знать кратчайшее расстояние до всех вершин-предшественников для целевой, зная эти расстояния, мы можем дальше выбрать самое короткое ребро из ведущих в целевую вершину и завершить поиск. Аналогично, чтобы найти кратчайшее расстояние до всех вершин предшественников нужно знать кратчайшее расстояние до их предшественников и так далее до самого начала обхода. Другими словами, нам нужно посчитать кратчайшие расстояния от начальной вершины до каждой вершины в графе.

Будем хранить эти расстояния в словаре {\firacodebold distances}, где ключем будет {\firacodebold id} текущей вершины, а значением минимальное {\bfseries известное} расстояние от нее до начальной. При этом, так как даже для промежуточных вершин может быть найден более короткий путь в процессе обхода, а не только для целевой, то алгоритм подмены более длинного пути на более короткий будет общим для всех вершин.

Мы не знаем наверняка, а не найдем ли более короткий путь, не обойдя весь граф, поэтому завершаем обход после того как обойден весь граф (массив {\firacodebold nextEdges} стал пустым).

Полный код этого алгоритма приведен в листинге \ref{lst:bfsShortestPath}.

% Тут будет объяснение как работает алгоритм

Теперь давайте более подробно изучим код. Первое что изменилось, это момент, в который мы начинаем восстанавливать путь. Так как теперь мы не имеем права досрочно завершать обход в момент когда мы нашли целевую вершину, то мы просто запоминаем ее в отдельную переменную {\firacodebold targetNode} (строки 12-14), как только найдем, а восстановление пути делаем на строках 42-50. Логика не добавления уже посещенных вершин также сохранилась (строки 36-38).

Оставшаяся часть алгоритма из листинга \ref{lst:bfsWeightedRandomPath} (выставление предшественника в поле {\firacodebold parent}) тоже присутствует, но видоизменилась. На строках с 20 по 34 мы обрабатываем каждое исходящее ребро {\firacodebold nextEdge} текущей вершины {\firacodebold currentNode}. В случае, если ребро ведет до еще ранее не просмотренной вершины (для нее нету значения в словаре {\firacodebold distance}, строки 31-34), считаем, что кратчайшее расстояние до этой вершины равно кратчайшему расстоянию до предшественника вершины {\firacodebold nextNode}, которым является вершина {\firacodebold nextNode}, плюс длина обрабатываемого ребра {\firacodebold nextEdge}. В противном случае (строки 21-28), у нас есть два пути, длины которых нам надо сравнить и выбрать кротчайший, другими словами, оставить все как есть, если новый найденный путь длиннее или равен текущему, и обновить путь, если он короче. Ключевым здесь является то, что мы не только обновили расстояние в словаре {\firacodebold distance}, но и поле {\firacodebold parent} для {\firacodebold nextNode}, ведь чтобы действительно прийти в nextNode по новому найденному кратчайшему пути, надо идти туда из текущей обрабатываемой вершины.

Как и прежде, ссылки {\firacodebold parent} формируют обратное дерево обхода, только теперь его форма будут зависеть не только от порядка обхода, как мы видели раньше для обходов в глубину и ширину, но и от весов ребер, так как мы будем "переставлять" ссылку {\firacodebold parent} в зависимости от того, какой {\firacodebold parent} предоставляет более короткий путь.

Раскадровка построения обратного дерева обхода для нашего случая приведена на рисунке \ref{fig:reverseTreeBfsShortestPath}.

\begin{figure*}
    \centering
    \begin{subfigure}{0.32\textwidth}
        \styledgraph{
            a⁰ -> [invisibleEdge, edge label=7] b [disabledNode],
            b -> [invisibleEdge, edge label=11, inner sep=1pt] c [disabledNode],
            b -> [invisibleEdge, edge label=2] d [disabledNode],
            d -> [invisibleEdge, edge label=1, inner sep=1.5pt] e [disabledNode],
            f [disabledNode] -> [invisibleEdge, edge label=7, inner sep=1.5pt] g [disabledNode],
            g -> [invisibleEdge, edge label=4] h [disabledNode],
            g -> [invisibleEdge, edge label=8] i [disabledNode],
            a⁰ -> [invisibleEdge, edge label=5] f,
        }
        \caption{Шаг 1}
        \label{fig:reverseTreeBfsShortestPath1}
        \vspace*{1cm}
    \end{subfigure}
    \begin{subfigure}{0.32\textwidth}
        \styledgraph{
            a⁰ <- [edge label=7] b⁷,
            b⁷ -> [invisibleEdge, edge label=11, inner sep=1pt] c [disabledNode],
            b⁷ -> [invisibleEdge, edge label=2] d [disabledNode],
            d -> [invisibleEdge, edge label=1, inner sep=1.5pt] e [disabledNode],
            f⁵ -> [invisibleEdge, edge label=7, inner sep=1.5pt] g [disabledNode],
            g -> [invisibleEdge, edge label=4] h¹⁰⁰,
            g -> [invisibleEdge, edge label=8] i [disabledNode],
            a⁰ <- [edge label=5] f⁵,
            h¹⁰⁰ -> [edge label=100] a⁰,
        }
        \caption{Шаг 2}
        \label{fig:reverseTreeBfsShortestPath2}
        \vspace*{1cm}
    \end{subfigure}
    \begin{subfigure}{0.32\textwidth}
        \styledgraph{
            a⁰ <- [edge label=7] b⁷,
            b⁷ <- [edge label=11, inner sep=1pt] c¹⁸,
            b⁷ <- [edge label=2] d⁹,
            d⁹ -> [invisibleEdge, edge label=1, inner sep=1.5pt] e [disabledNode],
            f⁵ -> [invisibleEdge, edge label=7, inner sep=1.5pt] g [disabledNode],
            g -> [invisibleEdge, edge label=4] h¹⁰⁰,
            g -> [invisibleEdge, edge label=8] i [disabledNode],
            a⁰ <- [edge label=5] f⁵,
            h¹⁰⁰ -> [edge label=100] a⁰,
        }
        \caption{Шаг 3}
        \label{fig:reverseTreeBfsShortestPath3}
        \vspace*{1cm}
    \end{subfigure}
    \begin{subfigure}{0.32\textwidth}
        \styledgraph{
            a⁰ <- [edge label=7] b⁷,
            b⁷ <- [edge label=11, inner sep=1pt] c¹⁸,
            b⁷ <- [edge label=2] d⁹,
            d⁹ -> [invisibleEdge, edge label=1, inner sep=1.5pt] e [disabledNode],
            f⁵ <- [edge label=7, inner sep=1.5pt] g¹²,
            g¹² -> [invisibleEdge, edge label=4] h¹⁰⁰,
            g¹² -> [invisibleEdge, edge label=8] i [disabledNode],
            a⁰ <- [edge label=5] f⁵,
            h¹⁰⁰ -> [edge label=100] a⁰,
        }
        \caption{Шаг 4}
        \label{fig:reverseTreeBfsShortestPath4}
    \end{subfigure}
    \begin{subfigure}{0.32\textwidth}
        \styledgraph{
            a⁰ <- [edge label=7] b⁷,
            b⁷ <- [edge label=11, inner sep=1pt] c¹⁸,
            b⁷ <- [edge label=2] d⁹,
            d⁹ <- [edge label=1, inner sep=1.5pt] c¹⁶,
            f⁵ <- [edge label=7, inner sep=1.5pt] g¹²,
            g¹² -> [invisibleEdge, edge label=4] h¹⁰⁰,
            g¹² -> [invisibleEdge, edge label=8] i [disabledNode],
            a⁰ <- [edge label=5] f⁵,
            h¹⁰⁰ -> [edge label=100] a⁰,
        }
        \caption{Шаг 5}
        \label{fig:reverseTreeBfsShortestPath5}
    \end{subfigure}
    \begin{subfigure}{0.32\textwidth}
        \styledgraph{
            a⁰ <- [edge label=7] b⁷,
            b⁷ <- [edge label=11, inner sep=1pt] c¹⁸,
            b⁷ <- [edge label=2] d⁹,
            d⁹ <- [edge label=1, inner sep=1.5pt] c¹⁶,
            f⁵ <- [edge label=7, inner sep=1.5pt] g¹²,
            g¹² <- [edge label=4] h¹⁶,
            g¹² <- [edge label=8] i²⁰,
            a⁰ <- [edge label=5] f⁵,
        }
        \caption{Шаг 6}
        \label{fig:reverseTreeBfsShortestPath6}
    \end{subfigure}
    \caption{Раскадровка обратного дерева обхода}
    \label{fig:reverseTreeBfsShortestPath}
\end{figure*}

С первого взгляда кажется, что все работает, мы действительно нашли кратчайший путь для нашего графа ({\firacodebold[a, f, g, h]} длиной 16). Но на самом деле алгоритм в общем случае использовать нельзя, так он не дает гарантированного результата.

Давайте добавим ровно одну вершину и два ребра в наш граф (рис. \ref{fig:exposeBfsError}) и посмотрим на шаг 4 алгоритма внимательнее (рис. \ref{fig:exposeBfsErrorStep}). Предположим что мы ищем путь от {\firacodebold a} к {\firacodebold j}.

\begin{figure}
    \styledgraph{
        a -> [edge label=7] b,
        b -> [edge label=11, inner sep=1pt] c,
        b -> [edge label=2] d,
        d -> [edge label=1, inner sep=1.5pt] e,
        f -> [edge label=7, inner sep=1.5pt] g,
        g -> [edge label=4, near start] h,
        g -> [edge label=8] i,
        a -> [edge label=5] f,
        i -> [bend right=45, edge label=10] f,
        c -> [bend left=45, edge label=5] a,
        h <- [edge label=100] a,
        h -> [edge label=2, inner sep=2.5pt] j,
        i -> [edge label=50, inner sep=2.5pt] j,
    }
    \caption{Еще один пример графа}
    \label{fig:exposeBfsError}
\end{figure}

\begin{figure*}
    \begin{subfigure}{0.32\textwidth}
        \styledgraph{
            a⁰ <- [edge label=7] b⁷,
            b⁷ <- [edge label=11, inner sep=1pt] c¹⁸,
            b⁷ <- [edge label=2] d⁹,
            d⁹ -> [invisibleEdge, edge label=1, inner sep=1.5pt] e [disabledNode],
            f⁵ -> [invisibleEdge, edge label=7, inner sep=1.5pt] g [disabledNode],
            g -> [invisibleEdge, edge label=4] h¹⁰⁰,
            g -> [invisibleEdge, edge label=8] i [disabledNode],
            a⁰ <- [edge label=5] f⁵,
            h¹⁰⁰ -> [edge label=100] a⁰,
            j¹⁰² -> [edge label=2, inner sep=2.5pt] h¹⁰⁰,
            i -> [invisibleEdge, edge label=50, inner sep=2.5pt] j¹⁰²,
        }
        \caption{Еще один пример графа}
        \label{fig:exposeBfsErrorStep}
    \end{subfigure}
    \begin{subfigure}{0.32\textwidth}
        \styledgraph{
            a⁰ <- [edge label=7] b⁷,
            b⁷ <- [edge label=11, inner sep=1pt] c¹⁸,
            b⁷ <- [edge label=2] d⁹,
            d⁹ <- [edge label=1, inner sep=1.5pt] c¹⁶,
            f⁵ <- [edge label=7, inner sep=1.5pt] g¹²,
            g¹² -> [invisibleEdge, edge label=4] h¹⁶,
            g¹² <- [edge label=8] i²⁰,
            a⁰ <- [edge label=5] f⁵,
            g¹² <- [edge label=4] h¹⁶,
            j¹⁰² [reverseNode] -> [reverseEdge, edge label=2, inner sep=2.5pt] h¹⁶,
            i²⁰ -> [invisibleEdge, edge label=50, inner sep=2.5pt] j¹⁰²,
        }
        \caption{Еще один пример графа}
        \label{fig:exposeBfsErrorStep2}
    \end{subfigure}
    \begin{subfigure}{0.32\textwidth}
        \styledgraph{
            a⁰ <- [edge label=7] b⁷,
            b⁷ <- [edge label=11, inner sep=1pt] c¹⁸,
            b⁷ <- [edge label=2] d⁹,
            d⁹ <- [edge label=1, inner sep=1.5pt] c¹⁶,
            f⁵ <- [edge label=7, inner sep=1.5pt] g¹²,
            g¹² -> [invisibleEdge, edge label=4] h¹⁶,
            g¹² <- [edge label=8] i²⁰,
            a⁰ <- [edge label=5] f⁵,
            g¹² <- [edge label=4] h¹⁶,
            j⁷⁰ [reverseNode] -> [invisibleEdge, edge label=2, inner sep=2.5pt] h¹⁶,
            i²⁰ <- [reverseEdge, edge label=50, inner sep=2.5pt] j⁷⁰,
        }
        \caption{Еще один пример графа}
        \label{fig:exposeBfsErrorStep3}
    \end{subfigure}
\end{figure*}


На рисунке \ref{fig:exposeBfsErrorStep} кратчайший и единственный путь от от {\firacodebold a} к {\firacodebold j} имеет длину 102. Если теперь перейти к шагу 6 (рис. \ref{fig:exposeBfsErrorStep2}) То мы увидим, что кратчайший путь до вершины {\firacodebold h} обновится, как и для предыдущего графа, но вот для верины {\firacodebold j} нет, так как мы не написали для это специально логики. То есть в словаре distance все еще записано 102 как кротчайший путь до вершины {\firacodebold j}. И если теперь перейти к последнему (седьмому) шагу то мы увидим картину на рисунке \ref{fig:exposeBfsErrorStep3}. Так как 102 (сохраненный кротчайший путь до вершины j) меньше чем 70 (текущий путь), то мы обновляем его в словаре distance и подменяем {\firacodebold parent}, в результате получая более длинный путь {\firacodebold [j, i, g, f, a]} = 70, чем могли бы ({\firacodebold [j, h, ,g, f, a]} = 18).

\subsection{Верный алгоритм}

Это последний шаг нашей трансформации алгоритма, когда он станет собственно алгоритмом Дейкстры, и будет лишен недостатка, описанного выше. И все что мы сделаем в этот раз, это изменим порядок обхода!

До этого мы реализовали обход в ширину, и как видно он не гарантирует правильного решения. Но если мы будем доставать из массива {\firacodebold nextNodes} (строка \ref{code:bfsShortestPath:unshift} в листинге \ref{lst:bfsShortestPath}) не самый первый элемент, а тот, который имеет минимальное расстояние до начальной вершины (соответствующие расстояния записаны в словаре {\firacodebold distances}), то мы получим алгоритм Дейкстры.

\begin{figure*}
    \begin{ruledelement}
        \lstset{
            emph={getNodeFromQueue, currentNode},
            emphstyle=\textbf,
        }
        \begin{lstlisting}[caption={Отличие алгоритма дейкстры}, label={lst:dijkstraDiffBfs}]
// take the shortest sub-path in the `nextNodes`
function getNodeFromQueue(nextNodes, distances) {
    let nextNodeId = null;
    let resultIndex = -1;

    for (let i = 0; i < nextNodes.length; i++) {
        if (distances.get(nextNodes[i].id) < distances.get(nextNodeId)) {
            nextNodeId = nextNodes[i].id;
            resultIndex = i;
        }
    }

    const [nextNode] = nextNodes.splice(resultIndex, 1);

    return nextNode;
}

// ...

// currentNode = nextEdges.shift();
currentNode = getNodeFromQueue(nextNodes, distances);
        \end{lstlisting}
    \end{ruledelement}
\end{figure*}

Аналогичное поведение достигается заменой массива nextNodes на очередь с приоритетом, но так в JavaScript нет готовой структуры для этого, проще, с точки зрения реализации, написать функцию вынимающую наиболее приоритетный элемент из обычного массива.

\begin{figure}
    \begin{tikzpicture}
    \begin{scope}[
        every node/.style={
            circle, draw, very thick, accentColor, minimum size=9mm,
            text=textColor, font=\bfseries,
            pattern=hatch, pattern color={accentColor!70}, hatch size=4pt,
        },
        reverseEdge/.style={secondaryAccentColor},
        reverseNode/.style={
            secondaryAccentColor,
            pattern=hatch, pattern color={secondaryAccentColor!70}, hatch size=4pt,
            text=textColor,
        },
        disabledEdge/.style={secondaryColor},
        disabledNode/.style={
            secondaryColor,
            pattern=hatch, pattern color={secondaryColor!70}, hatch size=4pt,
        },
        invisibleEdge/.style={backgroundColor, text=backgroundColor},
        invisibleNode/.style={backgroundColor, fill=backgroundColor, text=backgroundColor},
        clearNode/.style={draw=none, fill=none},
    ]
        \graph [
            layered layout,
            sibling distance=2cm,
            edges={
                shorten <=1pt,>={Stealth[round]},very thick,accentColor, inner sep=0.5pt, font=\bfseries,
            },
        ] {
            start -> { a -> [disabledEdge] b [disabledNode] -> [disabledEdge] c [disabledNode] -> [disabledEdge] d [disabledNode] -> [disabledEdge] end, end},
            start -> end;

            { [same layer] start, end };
            { [same layer] a, b, c, d };
        };
        \begin{scope}[node distance=10pt]
            \node [below=of d, minimum size=5pt] (abc) {};
            \node [right=of abc, minimum size=5pt, clearNode] {рассматриваемая};
            \node [below=of abc, minimum size=5pt, disabledNode] (bca) {};
            \node [right=of bca, minimum size=5pt, clearNode] {гипотетическая};
            \node [below=of bca, minimum size=5pt, reverseNode] (hhh) {};
            \node [right=of hhh, minimum size=5pt, clearNode] {активная};
        \end{scope}
    \end{scope}
    \end{tikzpicture}
    \caption{Пути во взвешенном графе от {\firacodebold a} к {\firacodebold h}}
    \label{fig:dijkstraProve}
\end{figure}

\subsection{Доказательство верности}

Мы уже увидели, что кажущийся с первого взгляда верный алгоритм, может так и не быть. К счастью, алгоритм Дейкстры имеет доказательство.

Прежде чем перейти к формальному доказательству, можно заметить, что проблема, описанная для первого алгоритма, показанного в листинге \ref{lst:bfsShortestPath}, заключалась в том, что мы попали в вершину h по более дилнному пути [a, h] раньше, чем по более короткому [a, f, g, h], и из-за того, что словарь distances перестал быть валидным хранилищем кратчайших расстояний от начальной вершины до любой другой обойденной вершины, а алгоритм Дейкстры исключает такую возможность. Если мы всегда попадаем в кратчайший путь от начальной вершины, до любой другой, то алгоритм работает верно.

Выбирая следующую вершину для обхода из nextNodes, мы по сути выбираем следующий путь, от которого мы продолжим анализ. Предположим, в какой-то момент алгоритм совершил ошибку и вначале попал в вершину end по более длинному пути PathY, чем тот, который еще не открыт (PathX). Это означало бы, что PathX включает в себя как минимум один из уже открытых путей., но все открытые пути по определению алгоритма должны быть длиннее PathY (на каждом этапе мы всегда выбираем самый короткий путь), значит PathX должен включать в себя путь, более длинный чем PathY, что означает, что PathX не может быть короче PathY. Следовательно, и совершение алгоритмом ошибки невозможно.

\end{document}
