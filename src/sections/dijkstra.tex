\documentclass[../article.tex]{subfiles}
\begin{document}

\sectionabstract{Алгоритм Дейкстры}
{Мы уже поняли как найти кратчайший путь в не взвешенном графе, но для взвешенного графа само понятие кратчайшего пути отличается, а для его поиска в очередной раз немного изменим нашу структуру и алгоритм.}

Взвешенный граф, это граф, ребра которого имеют вес, то есть некое число, ассоциированное с этим ребром. Кротчайшим путем называют путь, сумма весов всех ребер которого наименьшая из возможных. Иллюстрация этого на рисунке \ref{fig:compareWeightedPaths}.

\begin{figure}
    \styledgraph{
        a -> [disabledEdge] b [disabledNode],
        b -> c [disabledNode],
        b -> [disabledEdge] d [disabledNode],
        d -> [disabledEdge] e [disabledNode],
        f -> [edge label=7, inner sep=1.5pt] g,
        g -> [edge label=4, near start] h,
        g -> [disabledEdge] i [disabledNode],
        a -> [edge label=5] f,
        b -> [disabledEdge] c,
        i -> [bend right=45, disabledEdge] f,
        c -> [bend left=45, disabledEdge] a,
        h <- [edge label=100] a,
    }
    \caption{Пути во взвешенном графе от {\firacodebold a} к {\firacodebold h}}
    \label{fig:compareWeightedPaths}
\end{figure}

И хотя длина пути {\firacodebold [a, h]} в количестве ребер меньше, чем у {\firacodebold [a, f, g, h]}, все таки во взвешенном графе путь {\firacodebold [a, f, g, h]} считается короче.

\end{document}