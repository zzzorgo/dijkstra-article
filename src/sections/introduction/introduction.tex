\documentclass[../../article.tex]{subfiles}
\begin{document}

\def \pluralsightArticleUrl {https://www.pluralsight.com/content/dam/pluralsight2/landing-pages/offers/flow/pdf/Pluralsight\_20Patterns\_ebook.pdf}

\def \repourl {https://github.com/zzzorgo/dijkstra-article}

\sectionabstract{Введение}{В этой части вкратце расскажу зачем эта статья была написана. Если вас интересует только техническое содержание, ее можно смело пропустить}

Когда я в очередной раз понял, что пора менять работу, то как обычно начал готовиться к собеседованиям и столкнулся с проблемой: есть определенный набор тем, которые я бы хотел разобраться максимально быстро, но нет точечных материалов, эффективно заполняющих эти пробелы. Одна из таких тем это тот самый (как выяснилось, не сложный) алгоритм Дейкстры.

Конечно, я гуглил различные статьи и смотрел книги, которые могли бы его объяснить. Почему-то ни одно из найденных мной объяснений не отличалось понятностью. В какой-то момент я добрался до этого алгоритме в книге Стивена Скиены "Алгоритмы. Руководство по разработке". Книга написано крайне\ldots странно. Все примеры кода с однобуквенными переменными как будто специально обфусцированы, а объяснения хода выполнения не полные. Не хочу преуменьшить мастерства автора в написании алгоритмов, он явно в этом хорош, но книга вышла не самой простой.

В итоге, посидев над кодом в IDE и изрядно почитав самых разных объяснений, ко мне пришло понимание, которым я и хочу поделиться. Надеюсь мои страдания помогут сделать эту статью более понятной, чем остальные источники.

\subsection{Ссылки}

Стоит сказать, что за оформление статьи надо отдать должное прекрасно сверстанной статье от Pluralsight: \httplink{\pluralsightArticleUrl}{"20 patterns to watch for in your engineering team"}.

Сама статья написана с помощью \LaTeX и ее исходный код, а также все код из листингов можно найти у меня на \httplink{\repourl}{github}.

\end{document}
