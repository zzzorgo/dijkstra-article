\documentclass[../../article.tex]{subfiles}
\begin{document}

\sectionabstract{Введение}{В этой части вкратце расскажу зачем эта статья была написана. Если вас интересует только техническое содержание, ее можно смело пропустить}

Когда я в очередной раз понял, что пора менять работу то как обычно начал готовиться к собеседованиям и столкнулся с очевидной проблемой: есть определенный набор тем, которые я бы хотел разобраться максимально быстро. Одна из таких тем это тот самый (как выяснилось не сложный) алгоритм Дейкстры.

\subsection{Суб-секция}

Почему-то ни одно из найденных мной объяснений не отличалось понятностью. В итоге я добрался до этого алгоритме в книге Стивена Скиена "Алгоритмы. Руководство по разработке". Книга написано крайне... странно. Все примеры кода с однобуквенными переменными как будто специально обфусцированы, а объяснения хода выполнения не полные. Не хочу преуменьшить мастерства автора в написании алгоритмов, он явно в этом хорош, но книга вышла не самой простой.

\end{document}
