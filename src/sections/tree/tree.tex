\documentclass[../../article]{subfiles}
\begin{document}
% todo привести терминологию в порядок (вершина/элемент)
% todo заменить дефис на тире
\sectionabstract{Обход деревьев \label{tree}}
{Посмотрим, как выглядят деревья в коде, и на два вида их обхода: в глубину и в ширину. Если вы знакомы с идеей обхода деревьев и уверены, что не запутаетесь в трех соснах, можете смело переходить к следующей части.}

Обходят деревья, обычно, не просто так, а для поиска целевого элемента дерева. Именно поиском целевого элемента мы и займемся.

\subsection{Структура для обхода}

Прежде чем что-то обойти, нужно это что-то иметь. В данном случае нам нужна структура данных, которая представляет дерево. Наверняка, существует миллион разных способов представить древесную структура, но наша цель сделать это максимально просто и наглядно.

Пусть все дерево состоит из набора элементов типа {\firacodebold Node}, а корневой элемент лежит в переменной {\firacodebold treeRoot}.
% \begin{strip}
    \begin{ruledelement}
        \begin{lstlisting}[label={lst:treeStructure}]
interface Node {
    id: string;
    children: Node[];
}

const treeRoot: Node = // ...
        \end{lstlisting}
    \end{ruledelement}
% \end{strip}

Здесь {\firacodebold id} это уникальный идентификатор элемента (а по совместительству полезная нагрузка), а {\firacodebold children} – массив всех непосредственных потомков этого элемента. Каждый потомок это тоже элемент типа {\firacodebold Node} со своим {\firacodebold id} и потомками. Так и выстраивается структура дерева как на рисунке \ref{fig:exampleTree}.

\begin{figure}
    \styledgraph{
        treeRoot -- { / -- { / , / -- / }, / -- / -- { / , / }}
    }
    \caption{Пример дерева}
    \label{fig:exampleTree}
\end{figure}

Теперь, когда у нас есть структура данных для хранения дерева и корень дерева в переменной {\firacodebold treeRoot} мы можем приступить непосредственно к обходу.

Надо сказать, что оба способа обхода, которые мы рассмотрим, очень похожи, и буквально отличаются одной строчкой. Разница между ними заключается в том, в каком порядке будет происходить обход. Поэтому, от того, какой именно алгоритм был выбран может зависеть только то, как быстро мы найдем нужный элемент.

\begin{figure*}
    \lstset{
        emph={traverseTree},
        emphstyle=\textbf,
    }
    \begin{ruledelement}
        \begin{lstlisting}[caption={Обход дерева в глубину}, label={lst:treeDfs}]
function traverseTree() {
    let currentNode = null;
    const nextNodes = [treeRoot]; (*\label{code:treeDfs:stackInit}*)

    while (nextNodes.length !== 0) { (*\label{code:treeDfs:loop}*)
        currentNode = nextNodes.pop(); (*\label{code:treeDfs:stackPop}*)

        if (currentNode.id === 'h') { (*\label{code:treeDfs:checkForTarget}*)
            return currentNode;
        }

        nextNodes.push(...currentNode.children); (*\label{code:treeDfs:addChildren}*)
    }

    return null; (*\label{code:treeDfs:notFound}*)
}
        \end{lstlisting}
    \end{ruledelement}
\end{figure*}

\begin{figure}
    \styledgraph
    {
        1 -- { 6 -- { 9 , 7 -- 8 }, 2 -- 3 -- { 5 , 4 }}
    }
    \caption{Порядок обхода в глубину}
    \label{fig:treeDfsOrder}
\end{figure}

\subsection{Обход в глубину}

Обход происходит последовательно, в каждой итерации обхода мы исследуем ровно один элемент. Вопрос только в том, откуда мы его берем? Ну что ж, первый элемент у нас уже есть, это {\firacodebold treeRoot} или корень дерева. Исследовав его мы можем начать исследовать всех его потомков, а затем потомков этих потомков и так далее, пока не найдем требуемый элемент или не переберем все элементы.

Именно этот алгоритм и записан в листинге \ref{lst:treeDfs}. До начала выполнения основного цикла на строке \ref{code:treeDfs:loop} мы инициализируем массив {\firacodebold nextNodes} (строка \ref{code:treeDfs:stackInit}). Туда мы будем класть все элементы, которые мы хотим обойти. На данный момент у нас есть только корень дерева, его и кладем. Строга говоря, {\firacodebold nextNodes} в данном примере исполняет роль не столько массива, сколько {\firacodebold стека}.

Войдя в цикл, первым делом мы достаем верхний элемент стека (строка \ref{code:treeDfs:stackPop}) и кладем его в переменную {\firacodebold currentNode}. На этом этапе это {\firacodebold treeRoot}. Сразу после этого проверяем не является ли {\firacodebold currentNode} целевым элементом (строка \ref{code:treeDfs:checkForTarget}). В нашем случае целевой элемент, тот, у которого поле {\firacodebold data} равно строке {\firacodebold 'h'}.

Если это действительно целевой элемент – выходим из цикла и функции, обход завершен, целевой элемент найден.

\begin{figure*}
    \lstset{
        emph={shift},
        emphstyle=\textbf,
    }
    \begin{ruledelement}
        \begin{lstlisting}[caption={Обход дерева в ширину}, label={lst:treeBfs}]
function traverseTree() {
    let currentNode = null;
    const nextNodes = [treeRoot];

    while (nextNodes.length !== 0) {
        currentNode = nextNodes.shift(); (*\label{code:treeBfs:stackPop}*)

        if (currentNode.id === 'h') {
            return currentNode;
        }

        nextNodes.push(...currentNode.children);
    }

    return null;
}

        \end{lstlisting}
    \end{ruledelement}
\end{figure*}

\begin{figure}
    \styledgraph
    {
        1 -- { 2 -- { 4 , 5 -- 7 }, 3 -- 6 -- { 8 , 9 }}
    }
    \caption{Порядок обхода в ширину}
    \label{fig:treeBfsOrder}
\end{figure}

Если же нет, то добавляем на стек всех потомков текущего элемента {\firacodebold currentNode} (на данный момент это потомки {\firacodebold treeRoot}) и заходим на новую итерацию. Снова достаем верхний элемент из {\firacodebold nextNodes} и повторяем итерации до тех пока не найдем целевой элемент или не обойдем все дерево.

Если мы обойдем все дерево, так и не найдя целевой элемент, тогда в какой-то момент список {\firacodebold nextNodes} опустеет и мы выйдем из функции на строке \ref{code:treeDfs:notFound}.

Так почему же это поиск в глубину? На рисунке \ref{fig:treeDfsOrder} числами обозначен порядок обхода дерева. В этом алгоритме мы идем вглубь пока не дойдем до самого глубокого элемента и только после этого начинаем обрабатывать менее глубокие вершины, причем все еще максимально глубокие из доступных. Мы не перейдем ко второму потомку корневого элемента, пока не обойдем все поддерево первого потомка.

Такой порядок достигается за счет того, что {\firacodebold nextNodes} является стеком. На строке \ref{code:treeDfs:addChildren} мы кладем элементы в конец массива, а на строке \ref{code:treeDfs:stackPop} достаем из конца массива. То есть достаем всегда самых свеженьких потомков.

\subsection{Обход в ширину}

Если сравнивать обход в ширину с обходом в глубину, то у второго ровно одно отличие – вместо стека для {\firacodebold nextNodes} используется {\bfseries очередь}. В листинге \ref{lst:treeBfs} на строке \ref{code:treeBfs:stackPop} вместо последнего элемента массива мы достаем первый (это и делает наш массив очередью), а так как первыми в очередь попадают более близкие к корню потомки, то мы вначале обойдем их и только потом пойдем глубже. Порядок обхода показан на рисунке \ref{fig:treeBfsOrder}.

\subsection{Вместо заключения}
Наверняка вам сейчас захотелось попробовать этот код самим, и это отличная идея. Правда, помимо функции обхода дерева нужно еще и само дерево, формирование которого может быть утомительным. Поэтому ниже в листинге \ref{lst:treeJson} приведен пример дерева, которое можно скопировать к себе в код. Его структура соответствует той, что на рисунке \ref{fig:treeBfsOrder}.

\begin{ruledelement}
    \begin{lstlisting}[caption={Пример {\firacodebold treeRoot}}, label={lst:treeJson}]
const i = {
    data: 'i',
    children: [],
};
const h = {
    data: 'h',
    children: [],
};
const e = {
    data: 'e',
    children: [],
};
const g = {
    data: 'g',
    children: [h, i],
};
const d = {
    data: 'd',
    children: [e],
};
const c = {
    data: 'c',
    children: [],
};
const f = {
    data: 'f',
    children: [g],
};
const b = {
    data: 'b',
    children: [c, d],
};
const a = {
    data: 'a',
    children: [b, f],
};

const treeRoot = a;
    \end{lstlisting}
\end{ruledelement}

\end{document}
