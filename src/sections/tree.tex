\documentclass[../article.tex]{subfiles}
\begin{document}
% todo привести терминологию в порядок (вершина/элемент)
% todo заменить дефис на тире
\sectionabstract{Обход деревьев}
{Посмотрим, как выглядят деревья в коде, и на два вида их обхода: в глубину и в ширину. Если вы знакомы с идеей обхода деревьев и уверены, что не запутаетесь в трех соснах, можете смело переходить к следующей части.}

Обходят деревья, обычно, не просто так, а для поиска целевого элемента дерева. Именно поиском целевого элемента мы и займемся.

\subsection{Структура для обхода}

Прежде чем что-то обойти, нужно это что-то иметь. В данном случае нам нужна структура данных, которая представляет дерево. Наверняка, существует миллион разных способов представить древесную структура, но наша цель сделать это максимально просто и наглядно.

Пусть все дерево состоит из набора элементов типа {\firacodebold Node}, а корневой элемент лежит в переменной {\firacodebold treeRoot}.
% \begin{strip}
    \begin{ruledelement}
        \begin{lstlisting}[label={lst:treeStructure}]
interface Node {
  data: string;
  children: Node[];
}

const treeRoot: Node = // ...
        \end{lstlisting}
    \end{ruledelement}
% \end{strip}

Здесь {\firacodebold data} это некая полезная нагрузка некого элемента (деревья, как и любая другая структура данных должны что-то хранить), а {\firacodebold children} - массив всех непосредственных потомков этого элемента. Каждый потомок это тоже элемент типа {\firacodebold Node} со своими данными и потомками. Так и выстраивается структура дерева как на рисунке \ref{fig:exampleTree}.

\begin{figure}[t!]
    \centering
    \tikz \graph [
        tree layout,
        nodes={
            circle, draw, very thick, accentColor, minimum size=9mm,
            text=textColor, font=\bfseries,
            pattern=hatch, pattern color={accentColor!70}, hatch size=4pt
        },
        level distance=1.5cm,
        edges={shorten <=1pt,>={Stealth[round]},very thick,accentColor},
    ]
    {
        treeRoot -- { / -- { / , / -- / }, / -- / -- { / , / }}
    };
    \caption{Пример дерева}
    \label{fig:exampleTree}
\end{figure}


    \begin{ruledelement}
        \begin{lstlisting}[caption={Пример {\firacodebold treeRoot}}, label={lst:treeJson}]
const node9 = {
    data: '9',
    children: [],
};
const node8 = {
    data: '8',
    children: [],
};
const node7 = {
    data: '7',
    children: [],
};
const node6 = {
    data: '6',
    children: [node8, node9],
};
const node5 = {
    data: '5',
    children: [node7],
};
const node4 = {
    data: '4',
    children: [],
};
const node3 = {
    data: '3',
    children: [node6],
};
const node2 = {
    data: '2',
    children: [node4, node5],
};
const treeRoot = {
    data: '1',
    children: [node2, node3],
};
        \end{lstlisting}
    \end{ruledelement}


Теперь, когда у нас есть структура данных для хранения дерева и корень дерева в переменной {\firacodebold treeRoot} мы можем приступить непосредственно к обходу.

Надо сказать, что оба способа обхода очень похожи, и буквально отличаются одной строчкой. Разница между ними заключается в том, в каком порядке будет происходить обход. Поэтому, от того, какой именно алгоритм был выбран может зависеть только то, как быстро мы найдем нужный элемент.

\subsection{Обход в глубину}

Обход происходит последовательно, в каждой итерации обхода мы исследуем ровно один элемент. Вопрос только в том, откуда мы его берем? Ну что ж, первый элемент у нас уже есть, это {\firacodebold treeRoot} или корень дерева. Исследовав его мы можем начать исследовать всех его потомков, а затем потомков этих потомков и так далее, пока не найдем требуемый элемент или не переберем все элементы.

\begin{strip}
    \lstset{
        emph={traverseTree},
        emphstyle=\textbf,
    }
    \begin{topruledelement}
        \begin{lstlisting}[caption={Обход дерева в глубину}, label={lst:treeDfs}]
interface Node {
  data: string;
  children: Node[];
}

const treeRoot: Node = // ...

function traverseTree() {
    let currentNode = null;
    const nextNodes = [treeRoot];

    while (nextNodes.length !== 0) {
        currentNode = nextNodes.pop();

        if (currentNode.data === 'target data') {
            return currentNode;
        }

        nextNodes.push(...currentNode.children);
    }

    return null;
}

        \end{lstlisting}
    \end{topruledelement}
\end{strip}

Именно этот алгоритм и записан в листинге \ref{lst:treeDfs}. До начала выполнения основного цикла на строке 12 мы инициализируем массив {\firacodebold nextNodes} (стока 10). Туда мы будем класть все элементы, которые мы хотим обойти. На данный у нас есть только корень дерева, его и кладем. Строга говоря, {\firacodebold nextNodes} в данном примере исполняет роль не столько массива, сколько {\bfseries стека}.

Войдя в цикл, первым делом мы достаем верхний элемент стека (строка 13) и кладем его в переменную {\firacodebold currentNode}. На этом этапе это {\firacodebold treeRoot}. Сразу после этого проверяем не является ли {\firacodebold currentNode} целевым элементом (строка 15). В нашем случае целевой элемент, тот, у которого поле {\firacodebold data} равно строке {\firacodebold 'target data'}.

Если это действительно целевой элемент - выходим из цикла и функции, обход завершен, целевой элемент найден.

Если же нет, то добавляем на стек всех потомков текущего элемента (на данный момент это потомки {\firacodebold treeRoot}) и заходим на новую итерацию. Снова достаем верхний элемент из {\firacodebold nextNodes} и повторяем итерацию.

В худшем случае мы обойдем все дерево, так и не найдя целевой элемент, тогда в какой-то момент список {\firacodebold nextNodes} станет пустым и мы выйдем из функции на строке 22.

Так почему же это поиск в глубину? На рисунке \ref{fig:treeDfsOrder} числами обозначен порядок обхода дерева. В этом алгоритме мы идем вглубь пока не дойдем до самого глубокого элемента и только после этого начинаем обрабатывать менее глубокие вершины, причем все еще максимально глубокие из доступных. Мы не перейдем ко второму потомку корневого элемента, пока не обойдем все поддерево первого потомка.

Такой порядок достигается за счет того, что {\firacodebold nextNodes} является стеком. На строке 19 мы кладем элементы в конец массива, а на строке 13 достаем из конца массива. То есть достаем всегда самых свеженьких потомков.

\begin{figure}
    \centering
    \tikz \graph [
        tree layout,
        nodes={
            circle, draw, very thick, accentColor, minimum size=9mm,
            text=textColor, font=\bfseries,
            pattern=hatch, pattern color={accentColor!70}, hatch size=4pt
        },
        level distance=1.5cm,
        edges={shorten <=1pt,>={Stealth[round]},very thick,accentColor},
    ]
    {
        1 -- { 6 -- { 9 , 7 -- 8 }, 2 -- 3 -- { 5 , 4 }}
    };
    \caption{Порядок обхода в глубину}
    \label{fig:treeDfsOrder}
\end{figure}


\subsection{Обход в ширину}

Если сравнивать обход в ширину с обходом в глубину, то у второго ровно одно отличие - вместо стека для {\firacodebold nextNodes} используется {\bfseries очередь}. В листинге \ref{lst:treeBfs} на строке 13 вместо последнего элемента массива мы достаем первый (это и делает наш массив очередью), а так как первыми в очередь попадают более близкие к корню потомки, то мы вначале обойдем их и только потом пойдем глубже. Порядок обхода показан на рисунке \ref{fig:treeBfsOrder}.

\begin{figure}
    \centering
    \tikz \graph [
        tree layout,
        nodes={
            circle, draw, very thick, accentColor, minimum size=9mm,
            text=textColor, font=\bfseries,
            pattern=hatch, pattern color={accentColor!70}, hatch size=4pt
        },
        level distance=1.5cm,
        edges={shorten <=1pt,>={Stealth[round]},very thick,accentColor},
    ]
    {
        1 -- { 2 -- { 4 , 5 -- 7 }, 3 -- 6 -- { 8 , 9 }}
    };
    \caption{Порядок обхода в ширину}
    \label{fig:treeBfsOrder}
\end{figure}

\subsection{Вместо заключения}
Наверняка вам сейчас захотелось попробовать этот код самим, и это отличная идея. Правда, помимо функции обхода дерева нужно еще и само дерево, формирование которого может быть утомительным. Поэтому ниже в листинге \ref{lst:treeJson} приведен пример дерева, которое можно скопировать к себе в код. Его структура и нейминг соответствует той, что на рисунке \ref{fig:treeBfsOrder}.

% \twocolumnbagebreak

\begin{strip}
    \lstset{
        emph={shift},
        emphstyle=\textbf,
    }
    \begin{ruledelement}
        \begin{lstlisting}[caption={Обход дерева в ширину}, label={lst:treeBfs}]
interface Node {
  data: string;
  children: Node[];
}

const treeRoot: Node = // ...

function traverseTree() {
    let currentNode = null;
    const nextNodes = [treeRoot];

    while (nextNodes.length !== 0) {
        currentNode = nextNodes.shift();

        if (currentNode.data === 'target data') {
            return currentNode;
        }

        nextNodes.push(...currentNode.children);
    }

    return null;
}

        \end{lstlisting}
    \end{ruledelement}
\end{strip}
\end{document}