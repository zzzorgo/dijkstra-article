\documentclass[../../article.tex]{subfiles}
\begin{document}
% todo привести терминологию в порядок (вершина/элемент)
% todo заменить дефис на тире

\sectionabstract{Обход графов}
{Как и в случае с деревьями, графы обходят не просто так. Посмотрим, как найти целевую вершину в графе и проведем аналогию с обходом деревьев.}

\subsection{Структура для обхода}

В процессе того, как мы будем двигаться по этому разделу, структура, которую мы будем использовать для объявления графа, изменится. Но начнем с максимально похожей на ту, что мы использовали для деревьев.

\begin{ruledelement}
    \lstset{
        emph={startingNode, neighbors, visited},
        emphstyle=\textbf,
    }
    \begin{lstlisting}[caption={Структура вершины графа}, label={lst:graphStructureTreeLike}]
interface Node {
    id: string;
    neighbors: Node[];
    visited?: boolean;
}


const startingNode: Node = // ...
    \end{lstlisting}
\end{ruledelement}

Как видно из листинга \ref{lst:graphStructureTreeLike}, поменялось не очень много. В классическом понимании у графа нет корня ({\firacodebold rootNode}), но в наших задачах есть вершина {\firacodebold startingNode}, с которой мы начинаем обход. Также у элементов нет потомков, только соседние элементы {\firacodebold neighbors}. Тем не менее, если исключить переименовывание, это та же структура (про поле {\firacodebold visited} поговорим чуть позже), а, значит, и обойти ее можно так же, как дерево?

На самом деле, не совсем так. Из \httplink{https://en.wikipedia.org/wiki/Tree_(data_structure)}{определения~деревьев} следует ограничение на то, какие именно элементы (вершины) могут быть потомками для других элементов. То есть, например, вершина-потомок не может иметь в качестве своего потомка своего же родителя. Также вершина дерева не может иметь больше одного родителя. Наличие же таких связей в итоговом дереве автоматически превратит его в граф.

\begin{figure}
    \styledgraph{
        a -> b,
        b -> c,
        b -> d,
        d -> e,
        f -> g,
        g -> h,
        g -> i,
        a -> f,
        b -> c,
        i -> [bend right=45] f,
        c -> [bend left=45] a,
        h <- a,
    }
    \caption{Пример графа}
    \label{fig:treeTurningToGraph}
\end{figure}

На рисунке \ref{fig:treeTurningToGraph} хорошо видно, как это происходит при, например, добавлении трех связей (ребер) к уже знакомому нам дереву: от {\firacodebold c} к {\firacodebold a}, от {\firacodebold i} к {\firacodebold f} и от {\firacodebold a} к {\firacodebold h}. Также, на этом рисунке более явно указано направление ребер, где стрелка идет от элемента-родителя до элемента-потомка. Добавление этих ребер можно выразить в коде (листинг~\ref{lst:treeTurningToGraph}).

\begin{figure*}
    \lstset{
        emph={traverseGraph, visited, filter},
        emphstyle=\textbf,
    }
    \begin{ruledelement}
        \begin{lstlisting}[caption={Обход графа в глубину}, label={lst:graphDfs}]
function traverseGraph() {
    let currentNode = null;
    const nextNodes = [startingNode];

    while (nextNodes.length !== 0) {
        currentNode = nextNodes.pop();
        currentNode.visited = true; (*\label{code:graphDfs:setVisited}*)

        if (currentNode.id === 'h') {
            return currentNode;
        }

        const nodesToAdd = currentNode.neighbors (*\label{code:graphDfs:visitedUsed:start}*)
            .filter(nextNode => !nextNode.visited);

        nextNodes.push(...nodesToAdd); (*\label{code:graphDfs:visitedUsed:end}*)
    }

    return null;
}
        \end{lstlisting}
    \end{ruledelement}

\end{figure*}

\begin{figure}
    \begin{ruledelement}
        \lstset{
            emph={neighbors,push},
            emphstyle={\textbf},
            firstnumber=38,
        }
        \begin{lstlisting}[caption={Добавление связей превращающих дерево в граф}, label={lst:treeTurningToGraph}]
// previous tree declaration...

i.neighbors.push(f);
c.neighbors.push(a);
a.neighbors.push(h);

const startingNode = a;
        \end{lstlisting}
    \end{ruledelement}
\end{figure}

\begin{figure}
    \styledgraph{
        1 -> 2,
        2 -> 5,
        2 -> 6,
        6 -> 8,
        4 -> 7,
        7 -> 3,
        7 -> 9,
        1 -> 4,
        2 -> 5,
        9 -> [bend right=45] 4,
        5 -> [bend left=45] 1,
        3 <- 1,
    }
    \caption{Порядок обхода графа в ширину}
    \label{fig:graphBfsOrder}
\end{figure}

Что ж, после этих манипуляций структура дерева не сильно поменялась, как и алгоритм ее обхода, хоть это теперь и граф.

\subsection{Обход в глубину}

Предположим, что мы не стали бы менять наш алгоритм и подсунули ему наш граф (рис. \ref{fig:treeTurningToGraph}). В таком случае все работает так же, как и для дерева до определенного момента. Первым делом, наш обход в глубину пойдет по правой ветке графа и дойдет до вершины {\firacodebold f} на второй итерации. На четвертой итерации он дойдет до обработки вершины {\firacodebold i}. В момент обработки элемента {\firacodebold i} алгоритм добавит всех его потомков, то есть включая вершину {\firacodebold f}, в стек обработки {\firacodebold nextNodes}. То есть мы снова вернемся к обработке элемента {\firacodebold f} и снова добавим всех потомков {\firacodebold f} на обработку, а дальше потомков его потомков, включая вершину {\firacodebold i}. В итоге получается бесконечный цикл.

Если же мы исключим из обхода уже обработанные вершины, то второй раз в вершину {\firacodebold f} мы не попадем. В этом и есть все отличие (листинг \ref{lst:graphDfs}).

В момент обработки очередного элемента {\firacodebold currentNode} мы выставляем флаг {\firacodebold visited} в значение {\firacodebold true} (строка \ref{code:graphDfs:setVisited}), чтобы на последующих итерациях не добавлять обработанный элемент в стек {\firacodebold nextNodes} (строки \ref{code:graphDfs:visitedUsed:start}-\ref{code:graphDfs:visitedUsed:end}). Порядок обхода в данном случае не отличается от порядка обхода дерева из раздела \ref{tree} и его можно посмотреть на рисунке \ref{fig:treeDfsOrder} (хотя он мог поменяться, но для простоты мы выбрали такой граф, где этого не происходит).

\subsection{Обход в ширину}

Кажется довольно очевидным, что обход в ширину, как и для деревьев, будет отличаться лишь одной сточкой (листинг \ref{lst:graphBfsDiff}). Однако, из-за ребра {\firacodebold a → h} порядок обхода нашего графа в ширину изменится. На первой же итерации обхода алгоритм добавит всех потомков элемента {\firacodebold a} в очередь {\firacodebold nextNodes}, включая элемент {\firacodebold h}. А значит, он будет обработан одним из первых (рис. \ref{fig:graphBfsOrder}).

\begin{figure}
    \begin{ruledelement}
        \lstset{
            emph={shift},
            emphstyle=\textbf,
            firstnumber=8,
        }
        \begin{lstlisting}[caption={Отличие обхода в ширину от обхода в глубину}, label={lst:graphBfsDiff}]
// currentNode = nextNodes.pop();
currentNode = nextNodes.shift();
        \end{lstlisting}
    \end{ruledelement}
\end{figure}

\begin{figure}
    \styledgraph{
        a -> [disabledEdge] b [disabledNode],
        b -> c [disabledNode],
        b -> [disabledEdge] d [disabledNode],
        d -> [disabledEdge] e [disabledNode],
        f -> g,
        g -> h,
        g -> [disabledEdge] i [disabledNode],
        a -> f,
        b -> [disabledEdge] c,
        i -> [bend right=45, disabledEdge] f,
        c -> [bend left=45, disabledEdge] a,
        h <- a,
    }
    \caption{Пути в графе от {\firacodebold a} к {\firacodebold h}}
    \label{fig:pathExposure}
\end{figure}

\begin{figure}
    \begin{ruledelement}
        \lstset{
            emph={parent},
            emphstyle=\textbf,
        }
        \begin{lstlisting}[caption={Структура вершины графа для поиска пути}, label={lst:graphStructureSimplePath}]
interface Node {
    id: string;
    neighbors: Node[];
    visited?: boolean;
    parent?: Node;
}
        \end{lstlisting}
    \end{ruledelement}
\end{figure}

\subsection{Поиск пути}

Помимо поиска отдельно взятой вершины графа, часто бывает полезно найти еще и путь до этой вершины от той вершины, с которой мы начинали ({\firacodebold startingNode}). Но что такое найти путь? Будем считать, что путь – это массив вершин, первый элемент которого – это все та же {\firacodebold startingNode}, последний – целевая вершина, а в середине все промежуточные вершины, которые нужно будет пройти, чтобы дойти до целевой.

\begin{figure*}
    \lstset{
        emph={map, parent},
        emphstyle=\textbf,
    }
    \begin{ruledelement}
        \begin{lstlisting}[caption={Обход графа в глубину и формирование обратного дерева обхода}, label={lst:graphDfsParent}]
function traverseGraph() {
    let currentNode = null;
    const nextNodes = [startingNode];

    while (nextNodes.length !== 0) {
        currentNode = nextNodes.pop();
        currentNode.visited = true;

        if (currentNode.id === 'h') {
            return currentNode;
        }

        const nodesToAdd = currentNode.neighbors
            .filter(nextNode => !nextNode.visited)
            .map(nextNode => { (*\label{code:graphDfsParent:setParent:start}*)
                nextNode.parent = currentNode;

                return nextNode;
            }); (*\label{code:graphDfsParent:setParent:end}*)

        nextNodes.push(...nodesToAdd);
    }

    return null;
}
        \end{lstlisting}
    \end{ruledelement}

\end{figure*}

\begin{figure*}
    \centering
    \begin{subfigure}{0.32\textwidth}
        \styledgraph{
            a [disabledNode] <- [bend left=22] {b [disabledNode] <- [bend left=22] {c [disabledNode], d [disabledNode] <- [bend left=22] e [disabledNode]}, f [disabledNode] <- [bend left=22] g [disabledNode] <- [bend left=22] {h [disabledNode], i [disabledNode]}},

            a -> [disabledEdge] b [disabledNode],
            b -> c [disabledNode],
            b -> [disabledEdge] d [disabledNode],
            d -> [disabledEdge] e [disabledNode],
            f -> [disabledEdge] g,
            g -> [disabledEdge] h,
            g -> [disabledEdge] i [disabledNode],
            a -> [disabledEdge] f,
            b -> [disabledEdge] c,
            i -> [bend right=45, disabledEdge] f,
            c -> [bend left=45, disabledEdge] a,
            h <- [disabledEdge] a,
        }
        \caption{Связи {\firacodebold parent}}
        \label{fig:third}
        \label{fig:graphDfs:parents:edges}
    \end{subfigure}
    \begin{subfigure}{0.32\textwidth}
        \styledgraph{
            a  <-  {b  <-  {c , d  <-  e }, f  <-  g  <-  {h , i }},
        }
        \caption{Обратное дерево обхода}
        \label{fig:second}
        \label{fig:graphDfs:parents:reversedTree}
    \end{subfigure}
    \begin{subfigure}{0.32\textwidth}
        \styledgraph{
            a <- [disabledEdge] b [disabledNode],
            b <- c [disabledNode],
            b <- [disabledEdge] d [disabledNode],
            d <- [disabledEdge] e [disabledNode],
            f <- g,
            g <- h,
            g <- [disabledEdge] i [disabledNode],
            a <- f,
            b <- [disabledEdge] c,
        }
        \caption{Обратный путь}
        \label{fig:graphDfs:parents:reversedPath}
    \end{subfigure}
    \caption{Путь при обходе в глубину}
    \label{fig:graphDfs:parents}
\end{figure*}

\begin{figure*}
    \lstset{
        emph={parent},
        emphstyle=\textbf,
        firstnumber=11,
    }
    \begin{ruledelement}
        \begin{lstlisting}[caption={Построение пути до найденной вершины}, label={lst:graphPathConstruction}]
// if (currentNode.id === 'h') {
//     return currentNode;
// }
if (currentNode.id === 'h') {
    let pathNode = currentNode;
    const path = [];

    while(pathNode) {
        path.unshift(pathNode);
        pathNode = pathNode.parent;
    }

    return path;
}
        \end{lstlisting}
    \end{ruledelement}

\end{figure*}

Предположим, что мы ищем путь от вершины {\firacodebold a} до вершины {\firacodebold h} (рис. \ref{fig:pathExposure}). Видно, что есть ровно два возможных пути: {\firacodebold [a, h]} и {\firacodebold [a, f, g, h]}. Видно нам, но не алгоритму, так как для него входными данными является {\firacodebold startingNode}, а так как на каждой итерации мы обрабатываем ровно одну вершину, то после того, как мы нашли целевую вершину, у нас есть только конец и начало пути. Для некоторых путей, таких как {\firacodebold [a, h]} этого достаточно, но в общем случае, конечно же, нет. Нужно как-то получить весь путь.

Если подумать, то в процессе обхода мы приходим в любую вершину (кроме первой) от ее предшественника, и такой предшественник для вершины всегда один, так как мы не добавляем вершину в {\firacodebold nextNodes} больше одного раза. И именно это свойство можно использовать, сохраняя ссылку на предшественника. Для этого сохранения нам понадобится изменить изначальную структуру вершины (листинг \ref{lst:graphStructureSimplePath}), добавив в нее поле {\firacodebold parent} для хранения предшественника.

\begin{figure*}
    \centering
    \begin{subfigure}{0.32\textwidth}
        \styledgraph{

            a  [disabledNode] -> [disabledEdge] b [disabledNode],
            b -> c [disabledNode],
            b -> [disabledEdge] d [disabledNode],
            d -> [disabledEdge] e [disabledNode],
            f  [disabledNode] -> [disabledEdge] g [disabledNode],
            g -> [disabledEdge] h [disabledNode],
            g -> [disabledEdge] i [disabledNode],
            a -> [disabledEdge] f,
            b -> [disabledEdge] c,
            i -> [bend right=45, disabledEdge] f,
            c -> [bend left=45, disabledEdge] a,
            h <- [disabledEdge] a,

            h -> [bend left=10] a,
            i -> [bend right=22] g,
            g -> [bend right=22] f,
            f -> [bend right=22] a,
            e -> [bend right=22] d,
            d -> [bend right=22] b,
            c -> [bend right=22] b,
            b -> [bend right=22] a,
        }
        \caption{Связи {\firacodebold parent}}
        \label{fig:graphBfs:parents:edges}
    \end{subfigure}
    \begin{subfigure}{0.32\textwidth}
        \styledgraph{
            a  <-  {b  <-  {c , d  <-  e }, f  <-  g},
            g  <- [invisibleEdge] h,
            g  <- i,
            h -> a,
        }
        \caption{Обратное дерево обхода}
        \label{fig:graphBfs:parents:reversedTree}
    \end{subfigure}
    \begin{subfigure}{0.32\textwidth}
        \styledgraph{
            a <- [disabledEdge] b [disabledNode],
            b <- c [disabledNode],
            b <- [disabledEdge] d [disabledNode],
            d <- [disabledEdge] e [disabledNode],
            f [disabledNode] <- [disabledEdge] g [disabledNode],
            g <- [invisibleEdge] h,
            g <- [disabledEdge] i [disabledNode],
            a <- [disabledEdge] f,
            b <- [disabledEdge] c,
            h -> a,
        }
        \caption{Обратный путь}
        \label{fig:graphBfs:parents:reversedPath}
    \end{subfigure}
    \caption{Путь при обходе в ширину}
    \label{fig:graphBfs:parents}
\end{figure*}

Обойдя граф в глубину целиком (допустим, мы не нашли целевую вершину), мы добавим к исходному графу несколько связей {\firacodebold parent}, выделенных на рисунке \ref{fig:graphDfs:parents:edges}. Эти же связи вынесены в отдельную структуру (будем называть ее {\bfseries обратным деревом обхода}) на рисунке \ref{fig:graphDfs:parents:reversedTree}, а на рисунке \ref{fig:graphDfs:parents:reversedPath} выделена только интересующая нас часть этого обратного дерева обхода. Код, который необходимо добавить к обходу в глубину для формирования обратного дерева обхода, находится на листинге \ref{lst:graphDfsParent} на строках с \ref{code:graphDfsParent:setParent:start} по \ref{code:graphDfsParent:setParent:end}.

И все-таки, вернемся к поиску пути от вершины {\firacodebold a} до вершины {\firacodebold h}. Мы начали обходить наш граф в глубину, параллельно строить обратное дерево обхода, и в определенный момент дошли до целевой вершины {\firacodebold h}. На этом этапе у нас гарантированно есть выделенная на рисунке \ref{fig:graphDfs:parents:edges} часть обратного дерева обхода, так как именно от этих предшественников мы пришли в {\firacodebold h}. Более того, если взять вершину {\firacodebold h} и пройти по ссылкам {\firacodebold parent} вверх, мы получим искомый путь, только в обратном порядке (обратный путь). Изменение, которое необходимо для получения пути в прямом порядке, показано на листинге \ref{lst:graphPathConstruction}

Дерево обратного обхода существует и для обхода в ширину, как и восстановление пути, но из-за другого порядка обхода мы можем найти (и в данном примере найдем) совсем другой путь. На рисунке \ref{fig:graphBfs:parents:edges} выделены связи {\firacodebold parent}. В данном случае, от обхода в глубину отличие лишь в том, что теперь предшественником вершины {\firacodebold h} является вершина {\firacodebold a}. Отсюда немного другое дерево обратного обхода (рис. \ref{fig:graphBfs:parents:reversedTree}) и другой найденный путь (рис. \ref{fig:graphBfs:parents:reversedPath}).

Интересно то, что с точки зрения количества вершин, в найденном пути обход в ширину всегда гарантирует минимальный путь, так как он всегда вначале обходит самые близкие элементы к начальной вершине.

\end{document}
