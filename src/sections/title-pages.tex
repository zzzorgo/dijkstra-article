\begin{titlepage}
    \pagenumbering{gobble}
    \styledtitlepage{accentColor}{backgroundColor}
    \styledtitlepage{backgroundColor}{textColor}

    \newgeometry{textwidth=10cm, onecolumn, top=0cm, bottom=2.5cm}

    \vspace*{\fill}
    \tableofcontents
    \vspace*{\fill}

    \renewcommand{\abstractname}{}
    \begin{abstract}
        \def \pluralsightArticleUrl {https://www.pluralsight.com/content/dam/pluralsight2/landing-pages/offers/flow/pdf/Pluralsight\_20Patterns\_ebook.pdf}

        \def \repourl {https://github.com/zzzorgo/dijkstra-article}

        \newgeometry{textwidth=10cm, onecolumn, top=-2cm, bottom=2.5cm}

        \setlength{\parskip}{0.5cm}
        \setlength{\parindent}{0pt}
        \setlength{\columnsep}{1.3cm}
        \setlength{\voffset}{0pt}
        \noindent

        \vspace*{\fill}

        \section*{Введение}

        Когда я в очередной раз понял, что пора менять работу, то, как обычно, начал готовиться к собеседованиям и столкнулся с проблемой: есть определенный набор тем, в которых я бы хотел разобраться максимально быстро, но нет точечных материалов, эффективно заполняющих эти пробелы. Одна из таких тем - это тот самый (как выяснилось, несложный) алгоритм Дейкстры.

        Конечно, я гуглил различные статьи и смотрел книги, которые могли бы его объяснить. Почему-то ни одно из найденных мной объяснений не отличалось понятностью. В какой-то момент я добрался до этого алгоритме в книге Стивена Скиены "Алгоритмы. Руководство по разработке". Книга написано крайне... странно. Все примеры кода с однобуквенными переменными как будто специально обфусцированы, а объяснения хода выполнения неполные. Не хочу преуменьшить мастерство автора в написании алгоритмов, он явно в этом хорош, но книга вышла не самой простой.

        В итоге, посидев над кодом в IDE и изрядно почитав самых разных объяснений, я пришел к пониманию, которым и хочу поделиться. Надеюсь, мои страдания помогут сделать эту статью более понятной, чем остальные источники.

        \subsection*{Ссылки}

        Стоит сказать, что за оформление статьи надо отдать должное прекрасно сверстанной статье от Pluralsight: \httplink{\pluralsightArticleUrl}{"20 patterns to watch for in your engineering team"}.

        Сама статья написана с помощью {\LaTeX} и, ее исходный код, а также весь код из листингов можно найти у меня на \httplink{\repourl}{github}.

        \vspace*{\fill}
    \end{abstract}

    \restoregeometry
    \clearpage
    \pagenumbering{arabic}
\end{titlepage}
