\newgeometry{textwidth=10cm, onecolumn, bottom=2.5cm, top=0pt}
\setlength{\headwidth}{145mm}
\vspace*{\fill}
\section*{Заключение}

Если поискать слово "Дейкстра" на \httplink{https://habr.com}{хабре}, то мы увидим примерно 18 страниц результатов, по 20 статей на каждой (на момент написания статьи). Довольно трудно при этом претендовать на самую понятную статью, объясняющую этот алгоритм. Тем не менее, я попытался.

Если после прочтения у вас остались вопросы, или предложения, или угрозы, то буду рад их прочитать в любом месте, в котором вы готовы их оставить:

\begin{description}
    \item[]{\href{https://t.me/zzzorgo}{\bfseries \ul{Telegram:}} @zzzorgo}
    \item[]{\href{mailto:zzzorgo@gmail.com}{\bfseries \ul{Email:}} zzzorgo@gmail.com}
    \item[]{
        \httplink{https://github.com/zzzorgo/dijkstra-article/issues}
        {\bfseries Issues на github:} Для таких же педантов как я
    }
    \item[]{
        \httplink{https://github.com/zzzorgo/dijkstra-article/pulls}
        {\bfseries Pull Request на github:} Самый желанный способ, можно заодно добавить себя в список "Спасибо" на последней странице
    }
\end{description}

Отдельное спасибо, хочется сказать моей супруге за редактуру и терпение. Ну и нельзя забыть про \httplink{https://tex.stackexchange.com}{tex overflow} - бесконечный источник тайного знания про \LaTeX, и людей, готовых этим знанием поделиться.

\vspace*{\fill}
\restoregeometry
\clearpage
